\documentclass[11pt,reqno]{amsart}
\usepackage[a4paper]{geometry}
\usepackage{hyperref}
\usepackage{paralist}
\usepackage{geometry}
\usepackage[normalem]{ulem}
\usepackage{graphicx}
\usepackage{booktabs}
\usepackage{comment}
\usepackage{verbatim}
\usepackage{listings}
\usepackage{natbib}
%\usepackage{apacite}
\usepackage{color}
\usepackage{multicol}
\usepackage{enumitem}
\usepackage{listings}
\usepackage{tikz}
\usepackage{setspace}
\usepackage{pdflscape}
\providecommand{\e}[1]{\ensuremath{\times 10^{#1}}}

%\geometry{
%  body={6.9in, 10in},
%  left=0.8in,
%  top=0.75in
%}

%\geometry{letterpaper}
\setlength{\parskip}{12pt plus 10pt}

%\setstretch{1.0}
%\linespread{1.5}


\hypersetup{
	colorlinks,
	urlcolor=blue,
	citecolor = violet,
	citebordercolor = green,
	filebordercolor = red,
	linkbordercolor = blue,
	urlbordercolor={0 1 1}
	}
	


%\paperheight=11truein
%\paperwidth=8.5truein
%\textheight=9in
%\textwidth=6.5in

\parindent=0in


\newcommand{\headerstyle}{\textit {\bf}}
\newcommand{\titlestyle}{\textit {\bf}}
\newcommand{\sectionstyle}{\underline}
%\newcommand{\sectionline}{\underline{\hspace{\textwidth}}}
\newcommand{\jobstyle}{\textbf}{{\vspace{8pt}}
\newcommand{\groupstyle}{\textbf \textsc}

\newcommand{\itembf}{\item \textbf }
\linespread{1.6}

\title{Climate Modeling / S/SE Asian Monsoon Concept Note}
%\email{ald2187@columbia.edu}
\date{\today}
\author{Anthony Louis D'Agostino}
\begin{document}
\maketitle


\begin{abstract}
TITLE: Aerosol's Effects on the South Asian Summer Monsoon: Insights from GISS ModelE 

Output from the NASA GISS ModelE GCM (1850-2012) is analyzed to estimate the role aerosols play in monsoon onset and intensity.    Examine direct and indirect effects 

 Given anticipated reductions in aerosol emissions, the results prove useful for policymakers envisioning precipitation dynamics under a reduced aerosol emissions state.  

\end{abstract}


\section{Notes}

\begin{itemize}
\item Results from Paleoclimate Modeling Intercomparison Project (6kya) - Joussaume et al. 1999 
\item Examples where vegetation and surface water storage feedbacks have been included in model runs?  

\item Regional Climate Model Intercomparison Project - validation data sources: NCEP-NCAR reanalysis, Xie and Arkin 1997, SLP from Japan MET 


\item From AR5
	\begin{itemize}
		\item Weakening of SASM - in terms of meridional reach?? 
		\item Figures 14.4, 14.5 - representations of climate indices for various monsoonal regions 
		\item CMIP5 ensemble results associated with medium confidence 
	\end{itemize}


\item RCPs
	\begin{itemize}
		\item  
	\end{itemize}


\section{Questions}
\item What is the temporal resolution of these models? 	
	
	



\section{Background}


\section{Objectives}
\item Identify how well the models are able to reproduce current monsoonal patterns (identify what exactly this will require) 

\item Joussaume et al. 1999 
	\begin{itemize}
	\item Subgrid-scale parameterizations expectedly having a big effect on monsoon dynamics - what specifically is changing from model to model?  
	
	
	\end{itemize}


\section{Data Presentation}
\item See Fig. 2 from RCMIP paper in BAMS 2005 for depiction of temperature biases across models across locations 

\item X







\section{Data Sources}
\item NOAA Patmos-X  \href{http://cimss.ssec.wisc.edu/patmosx/}{http://cimss.ssec.wisc.edu/patmosx/} daily, monthly netCDFs - nothing compiled for longer time-scales
\item \href{http://gdata1.sci.gsfc.nasa.gov/daac-bin/G3/gui.cgi?instance_id=MODIS_DAILY_L3&gsid=MODIS_DAILY_L3_160.39.231.134_1402350812&selectedWSID=140235120431683&app=latlonplot_diff&selectedMap=}{Giovanni} products through MODIS 


\section{Introduction}

Brief discussion about direct and indirect effects
Discussion about relevance of monsoon to South Asian populations, particularly in agriculture 
Anticipate a reduction in aerosol concentrations as fuel choice turns towards options with better pollution control options.  
Levy et al. 2008 for an increase in BC in time -- 

The physics of aerosols' effect on the global climate is well understood (???) and can be categorized into both direct and indirect effects.  What is less understood is XXXX.  This is further compounded by uncertainty regarding the aerosols emission pathways of industrializing countries whose emission profiles are strongly tied to fuels used in power generation, and therefore subject to significant change in coming decades as large investments are made in new capacity (SOURCES).  

Aerosol composition is determined by XXXX 



Model E2 is described in detail in Miller et al. 2014 XXXXX   [Need to know more about how aerosols are treated in the model]    Direct, indirect, and total aerosol forcings for the South Asian region under NINT and TCADI physics show slight differences, with TCADI indicating more positive forcings under all three configurations.  


Good background in Rotstayn regarding fast and slow effects of aerosol unmasking 
They identify the value of single-forcing runs, though acknowledge that such are not a priority for the modeling community.  They use the CSIRO Mk3.6 (Mark 3.6) R21 (reduced horizontal resolution 5:6, 3:2) 

Identify the extent to which aerosols mask the effect of GHGs (following in Rotstayn et al. 2012) -- this provides information about expected effects if aerosol emissions abruptly drop.  



Single-forcing run examples: Roystayn et al. 2012 (changes in precip in NW Australia induced by aerosols) 

\end{itemize}




%\bibliographystyle{aer}
%\bibliography{SAsianAerosols}

\end{document}  

