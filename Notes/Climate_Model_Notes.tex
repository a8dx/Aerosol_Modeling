\documentclass[11pt,reqno]{amsart}
\usepackage[a4paper]{geometry}
\usepackage{hyperref}
\usepackage{paralist}
\usepackage{geometry}
\usepackage[normalem]{ulem}
\usepackage{graphicx}
\usepackage{booktabs}
\usepackage{comment}
\usepackage{verbatim}
\usepackage{listings}
\usepackage{natbib}
%\usepackage{apacite}
\usepackage{color}
\usepackage{multicol}
\usepackage{enumitem}
\usepackage{listings}
\usepackage{tikz}
\usepackage{setspace}
\usepackage{pdflscape}
\providecommand{\e}[1]{\ensuremath{\times 10^{#1}}}

%\geometry{
%  body={6.9in, 10in},
%  left=0.8in,
%  top=0.75in
%}

%\geometry{letterpaper}
\setlength{\parskip}{12pt plus 10pt}

%\setstretch{1.0}
%\linespread{1.5}


\hypersetup{
	colorlinks,
	urlcolor=blue,
	citecolor = violet,
	citebordercolor = green,
	filebordercolor = red,
	linkbordercolor = blue,
	urlbordercolor={0 1 1}
	}
	


%\paperheight=11truein
%\paperwidth=8.5truein
%\textheight=9in
%\textwidth=6.5in

\parindent=0in


\newcommand{\headerstyle}{\textit {\bf}}
\newcommand{\titlestyle}{\textit {\bf}}
\newcommand{\sectionstyle}{\underline}
%\newcommand{\sectionline}{\underline{\hspace{\textwidth}}}
\newcommand{\jobstyle}{\textbf}{{\vspace{8pt}}
\newcommand{\groupstyle}{\textbf \textsc}

\newcommand{\itembf}{\item \textbf }
\linespread{1.6}

\title{Climate Modeling / S/SE Asian Monsoon Concept Note}
%\email{ald2187@columbia.edu}
\date{\today}
\author{Anthony Louis D'Agostino}
\begin{document}
\maketitle


\begin{abstract}
TITLE: Moist Static Energy's Role in South Asian Summer Monsoon Dynamics

The South Asian Summer Monsoon (SASM) significantly impacts several hundred million people with livelihoods tied to the monsoon rains.  While key characteristics of a monsoon including onset, retreat, and Using output from daily-resolution Coupled Models Intercomparison Project Phase 5 (CMIP5) general circulation models across RCP scenarios, we examine MSE's role in monsoon onset under a 21st century context with strong radiative forcings.           We then compare result against those observed for 


Output from daily-resolution Coupled Models Intercomparison Project Phase 5 (CMIP5) general circulation models is used to assess moist static energy (MSE) as an indicator in characterizing the onset, time-mean, and retreat dimensions of the South Asian Summer Monsoon.  We examine results from multiple representative concentration pathways (RCP 4.5, RCP 8.5), as well as an abrupt 4x of CO_2 scenario to identify how changes in radiative forcing impact the MSE-monsoon relationship.  We discuss results in light of earlier work examining MSE as an indicator for the Sahelian monsoon.    


Output from the NASA GISS ModelE GCM (1850-2012) is analyzed to estimate the role aerosols play in monsoon onset and intensity.    Examine direct and indirect effects 



 Given anticipated reductions in aerosol emissions, the results prove useful for policymakers envisioning precipitation dynamics under a reduced aerosol emissions state.  

\end{abstract}




\section{Poster Text}

Moist Static Energy's Role in South Asian Summer Monsoon Dynamics


\subsection{Research Question}

Climate change is anticipated to alter several of the key dimensions by which we understand the South Asian Summer Monsoon (SASM), with current findings suggesting that ONSET DATE, INTENSITY, RETREAT DATE --- (references).  

Current representations of the Indian Summer monsoon are based on precipitation observations alone, therefore leading to noisy signals in determining onset and retreat dates.  Moist static energy is a potential alternative with cleaner intraseasonal dynamics based on earlier work.  

\subsection{Research Approach}

Show how MSE is calculated --   MSE as useful in predicting Sahelian rainfall (Bernard 1999) 

Miller making link between rainfall and MSE in Sahel - usefulnes of MSE and forecasting rainfall 

We use daily-resolution CMIP5 model output from NASA GISS and HadleyGEM2 under high-GHG forcing experiments (1\% annual increase, abrupt 4x-CO$_2$, and RCP 8.5) to investigate the MSE-precipitation relationship in climate states markedly different from present time.  This enables us to contrast fast and slow responses given the nature of the forcings introduction.  

Specifics about all the models working with here:  

Focus on surface response to forcings 

\subsection{Findings}


\subsection{Conclusion}


\subsection{References}





Discuss (coupling/decoupling of MSE and precip dynamics given some )  







\section{Background}

Monsoon forecasts and analyses have typically been predicated on precipitation levels and therefore subject to the noisiness of a signal stochastic in nature.  Schneider and Bordoni use an idealized GCM and find that the land-sea contrast is not essential for reproduction of monsoonal regime shifting (??), however eddy momentum fluxes are.  



\section{Monsoon Forecasting}

	\begin{itemize}
		\item This is where information from AR5 and related studies is included to discuss the future evolution of the monsoon 
	\end{itemize}


\section{Methdology}
	\begin{itemize}
		\item 
	\end{itemize}



\section{Notes}

\begin{itemize}
\item Results from Paleoclimate Modeling Intercomparison Project (6kya) - Joussaume et al. 1999 
\item Examples where vegetation and surface water storage feedbacks have been included in model runs?  

\item Regional Climate Model Intercomparison Project - validation data sources: NCEP-NCAR reanalysis, Xie and Arkin 1997, SLP from Japan MET 


\item From AR5 - Climate Phenomena and their Relevance for Future Regional Climate Change
	\begin{itemize}
		\item Weakening of SASM - in terms of meridional reach?? 
		\item Figures 14.4, 14.5 - representations of climate indices for various monsoonal regions 
		\item CMIP5 ensemble results associated with medium confidence 
		\item Section 9.5.2.4 for details on the range in model ability to reproduce the monsoon 
		\item ``'Active/break spells of the monsoon are related to the MJO a phenomenon that models simulate poorly''
		\item X
	\end{itemize}


\item Monsoon indicators and variables of interest
	\begin{itemize}
		\item Global monsoon precipitation intensity (GMI)
		\item Global monsoon area (GMA) - distinction between global land monsoon domain and ocean domain 
		\item DUR - monsoon durations (days) 
		\item All India Rainfall (AIR) - land-only gridpoints over 65-95E, 7-30N (Sperber et al. 2014)
		\item Indian Monsoon Circulation Index (Figure 14.5) : meridional diffrence of JJA 850hPa zonal winds averaged over 5-15N, 40-80E and 20-30N, 60-90E - likely to decrease in 21st century (based on all forcings - can construct this using single forcing) what are typical values of these wind speeds?
		\item Sperber et al. - comparison of CMIP5 models against CMIP3 in ability to reproduce a range of monsoon-related indicators: 850 hPa wind, time-mean rainfall, timing of monsoon onset, peak, duration and withdrawal: 40-160E, 20S-50N; latitude-time diagram averaged between 70-90E; onset defined as first pentad meeting/exceeding 5mmday-1, should relative  rainfall exceed 5mmday-1 for May-September (subtract out Jan. avg? Sec. 4.2 for details); diagnostic tests were selected out of work from the CLIVAR Asian-Australian Monsoon Panel Diagnostics Task Team 
		\item Other elements of the monsoon to be assessed: scope for additional analysis of other aspects of monsoon variability and change (e.g., Zhou et al. 2009c; Zhou and Zou 2010; Boo et al. 2011, Li and Zhou 2011; Meehl et al. 2012).  trend analysis absent (Ramesh and Goswami)
		\item Ramesh and Goswami 2014 - examination of CMIP5 models under different RCPs and results from monsoon indicators; continental India (70-85E, 5-30N) - use a `hierarchical' approach to evalute certain criteria/metrics as more important than others in evaluating model performance; D1-D4 (pg. 3) as bounding box for continental India, D5 for continental and oceans; express trends as \% of respective SDs for period 
		\item Lee and Wang 2012 - future change of global monsoon in CMIP5 -  ***
		\item Zhou, Wu, Wang 2009 - How well do AGCM capture monsoon structure; `monsoon year' spans summer of year 0 JJA(0), to spring of following year MAM(1) *** 
		\item Zhou et al. - CLIVAR C20C Project - review of previous models' ability to reproduce rainfall anomalies when SST is prescribed under an AGCM; A-AM domain is 40-160E, 30S-40N (as per Wang 2006); focus on 1950-1999; 1. The Webster–Yang index (WYI), which is defined as the vertical zonal wind shear between 200 and 850 hPa 4. (U850–U200) averaged over the South Asian region (0°–20°N, 40°–110°E) to measure the broad-scale South Asian summer (JJA) monsoon circulation anomalies (Webster and Yang 1992), 2. The Indian summer (JJA) monsoon index (IMI), which is defined as the meridional differences of the 850 hPa zonal winds averaged over the domains (5°–15°N, 40°–80°E) and (20°–30°N, 60°–90°E), that is, 5. IMI 1⁄4 U850ð5􏰔􏰺15􏰔N; 40􏰔􏰺80􏰔EÞ 􏰺 U850ð20􏰔􏰺30􏰔N; 60􏰔􏰺90􏰔EÞ  (1054); looked at monsoon weakening across models (based on index); weakening of monsoon-ENSO connection (Section 4) 
		\item Kumar et al. 1999 - On the weakening relationship; monsoon anomalies predicted by Himalayan snow cover, ENSO cycle; corr. between monsoon rainfall and 200-hPa velocity potential (surrogate for Walker circulation); 
		\item Cook et al. - Science 2010 - Asian Monsoon Failure -  NOT FINISHED
		\item Gautam, et al. GA - Aerosol and rainfall variability - makes distinction between early summer and late summer when examining aerosol effects; aerosol profile by month; 





		\item Possible targets: ENSO-monsoon teleconnection (Lau and Nath 2000; Turner et al. 2005); interannual vs. decadal timescales; examining tightness of ENSO-monsoon connection under different RCP levels;   

		\item Additional papers: Webster and Yang 1992 -- basis for WY Index (South Asian region); 

	\end{itemize}



	\item MSE as alternative to measuring precip for characterizing monsoon - correlation btw ENSO and monsoon 
			IPCC about teleconnections; TRMM?  soil moisture and ET? Ron: asking Gavin for run time and Adam S. - MSE as monsoon indicator; 

			onset date and MSE - using Hadley, onset date for RCP8.5 



\section{Questions}

\begin{itemize}
	\item Wind convergence (in the lower troposphere)

\end{itemize}	 	




\section{Possible Tasks}
	\begin{itemize}
		\item Run the Monsoon Circulation Index using both single-forcing runs and 1\% and 4x-CO2 abrupt - can identify the spatial extent ??? 
	\end{itemize}




	
\section{Meeting with Ron - Oct 17}
	Contour plot 
	MSE as contour - year on x, day on y - is there a tilt in the contours?  
	do this for MSE and for precip. - does this look cleaner under MSE vs. precip?   - first for grid cells and then zonal average?  
	Abrupt transitionn with the monsoon in regards rainfall (observed for some period, then turning off)
	Take a look at places with similar monsoon onset date 
	Removing cyclical components of MSE 
	How to remove noise without throwing out abrupt transitions - what filter does this? 
	Is it actually coming on abruptly?  
	Filter for extracting square wave frequency?   
	ABRUPT SEASONAL MIGRATION of itcz 
	Zonal avg of precip over day of year for subcontinent 
	Bordon Tapio Schneier NatGeo Science 
	Using loess filter vs. Fourier decomp.  
	Try spatial average - 
	Monsoon can be reproduced wihtou ocean-land T contrast - more of an inter-hemispheric heat transfer story 
	Least squares fit to square wave, recovering time and magnitude parameters 
	for t between t0 and t1, magnitude A, for time between t1 and t2, magnitude B -- 
	Look at each year individually, 3 magnitude values (before during and after transition) 
	Then over all observations, trends on parameters
	Ultimately, how robust are these definitions of onset?  






\section{Background}


\section{Objectives}
\item Identify how well the models are able to reproduce current monsoonal patterns (identify what exactly this will require) 

\item Joussaume et al. 1999 
	\begin{itemize}
	\item Subgrid-scale parameterizations expectedly having a big effect on monsoon dynamics - what specifically is changing from model to model?  
	
	
	\end{itemize}


\section{Data Presentation}
\item See Fig. 2 from RCMIP paper in BAMS 2005 for depiction of temperature biases across models across locations 

\item X







\section{Data Sources}
\item NOAA Patmos-X  \href{http://cimss.ssec.wisc.edu/patmosx/}{http://cimss.ssec.wisc.edu/patmosx/} daily, monthly netCDFs - nothing compiled for longer time-scales
\item \href{http://gdata1.sci.gsfc.nasa.gov/daac-bin/G3/gui.cgi?instance_id=MODIS_DAILY_L3&gsid=MODIS_DAILY_L3_160.39.231.134_1402350812&selectedWSID=140235120431683&app=latlonplot_diff&selectedMap=}{Giovanni} products through MODIS 


\section{Introduction}

Brief discussion about direct and indirect effects
Discussion about relevance of monsoon to South Asian populations, particularly in agriculture 
Anticipate a reduction in aerosol concentrations as fuel choice turns towards options with better pollution control options.  
Levy et al. 2008 for an increase in BC in time -- 

The physics of aerosols' effect on the global climate is well understood (???) and can be categorized into both direct and indirect effects.  What is less understood is XXXX.  This is further compounded by uncertainty regarding the aerosols emission pathways of industrializing countries whose emission profiles are strongly tied to fuels used in power generation, and therefore subject to significant change in coming decades as large investments are made in new capacity (SOURCES).  

Aerosol composition is determined by XXXX 



Model E2 is described in detail in Miller et al. 2014 XXXXX   [Need to know more about how aerosols are treated in the model]    Direct, indirect, and total aerosol forcings for the South Asian region under NINT and TCADI physics show slight differences, with TCADI indicating more positive forcings under all three configurations.  


Good background in Rotstayn regarding fast and slow effects of aerosol unmasking 
They identify the value of single-forcing runs, though acknowledge that such are not a priority for the modeling community.  They use the CSIRO Mk3.6 (Mark 3.6) R21 (reduced horizontal resolution 5:6, 3:2) 

Identify the extent to which aerosols mask the effect of GHGs (following in Rotstayn et al. 2012) -- this provides information about expected effects if aerosol emissions abruptly drop.  



Single-forcing run examples: Roystayn et al. 2012 (changes in precip in NW Australia induced by aerosols) 



\subsection{Terminology Guide}
	\item Advection: quantities are advected - e.g., enthaply, pollution - via a fluid's bulk motion




\end{itemize}




%\bibliographystyle{aer}
%\bibliography{SAsianAerosols}

\end{document}  

